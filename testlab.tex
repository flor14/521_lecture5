% Options for packages loaded elsewhere
\PassOptionsToPackage{unicode}{hyperref}
\PassOptionsToPackage{hyphens}{url}
%
\documentclass[
]{article}
\title{Lab 3 - Introduction to RStudio and R Markdown for reports and
presentations}
\author{}
\date{\vspace{-2.5em}}

\usepackage{amsmath,amssymb}
\usepackage{lmodern}
\usepackage{iftex}
\ifPDFTeX
  \usepackage[T1]{fontenc}
  \usepackage[utf8]{inputenc}
  \usepackage{textcomp} % provide euro and other symbols
\else % if luatex or xetex
  \usepackage{unicode-math}
  \defaultfontfeatures{Scale=MatchLowercase}
  \defaultfontfeatures[\rmfamily]{Ligatures=TeX,Scale=1}
\fi
% Use upquote if available, for straight quotes in verbatim environments
\IfFileExists{upquote.sty}{\usepackage{upquote}}{}
\IfFileExists{microtype.sty}{% use microtype if available
  \usepackage[]{microtype}
  \UseMicrotypeSet[protrusion]{basicmath} % disable protrusion for tt fonts
}{}
\makeatletter
\@ifundefined{KOMAClassName}{% if non-KOMA class
  \IfFileExists{parskip.sty}{%
    \usepackage{parskip}
  }{% else
    \setlength{\parindent}{0pt}
    \setlength{\parskip}{6pt plus 2pt minus 1pt}}
}{% if KOMA class
  \KOMAoptions{parskip=half}}
\makeatother
\usepackage{xcolor}
\IfFileExists{xurl.sty}{\usepackage{xurl}}{} % add URL line breaks if available
\IfFileExists{bookmark.sty}{\usepackage{bookmark}}{\usepackage{hyperref}}
\hypersetup{
  pdftitle={Lab 3 - Introduction to RStudio and R Markdown for reports and presentations},
  hidelinks,
  pdfcreator={LaTeX via pandoc}}
\urlstyle{same} % disable monospaced font for URLs
\usepackage[margin=1in]{geometry}
\usepackage{color}
\usepackage{fancyvrb}
\newcommand{\VerbBar}{|}
\newcommand{\VERB}{\Verb[commandchars=\\\{\}]}
\DefineVerbatimEnvironment{Highlighting}{Verbatim}{commandchars=\\\{\}}
% Add ',fontsize=\small' for more characters per line
\usepackage{framed}
\definecolor{shadecolor}{RGB}{248,248,248}
\newenvironment{Shaded}{\begin{snugshade}}{\end{snugshade}}
\newcommand{\AlertTok}[1]{\textcolor[rgb]{0.94,0.16,0.16}{#1}}
\newcommand{\AnnotationTok}[1]{\textcolor[rgb]{0.56,0.35,0.01}{\textbf{\textit{#1}}}}
\newcommand{\AttributeTok}[1]{\textcolor[rgb]{0.77,0.63,0.00}{#1}}
\newcommand{\BaseNTok}[1]{\textcolor[rgb]{0.00,0.00,0.81}{#1}}
\newcommand{\BuiltInTok}[1]{#1}
\newcommand{\CharTok}[1]{\textcolor[rgb]{0.31,0.60,0.02}{#1}}
\newcommand{\CommentTok}[1]{\textcolor[rgb]{0.56,0.35,0.01}{\textit{#1}}}
\newcommand{\CommentVarTok}[1]{\textcolor[rgb]{0.56,0.35,0.01}{\textbf{\textit{#1}}}}
\newcommand{\ConstantTok}[1]{\textcolor[rgb]{0.00,0.00,0.00}{#1}}
\newcommand{\ControlFlowTok}[1]{\textcolor[rgb]{0.13,0.29,0.53}{\textbf{#1}}}
\newcommand{\DataTypeTok}[1]{\textcolor[rgb]{0.13,0.29,0.53}{#1}}
\newcommand{\DecValTok}[1]{\textcolor[rgb]{0.00,0.00,0.81}{#1}}
\newcommand{\DocumentationTok}[1]{\textcolor[rgb]{0.56,0.35,0.01}{\textbf{\textit{#1}}}}
\newcommand{\ErrorTok}[1]{\textcolor[rgb]{0.64,0.00,0.00}{\textbf{#1}}}
\newcommand{\ExtensionTok}[1]{#1}
\newcommand{\FloatTok}[1]{\textcolor[rgb]{0.00,0.00,0.81}{#1}}
\newcommand{\FunctionTok}[1]{\textcolor[rgb]{0.00,0.00,0.00}{#1}}
\newcommand{\ImportTok}[1]{#1}
\newcommand{\InformationTok}[1]{\textcolor[rgb]{0.56,0.35,0.01}{\textbf{\textit{#1}}}}
\newcommand{\KeywordTok}[1]{\textcolor[rgb]{0.13,0.29,0.53}{\textbf{#1}}}
\newcommand{\NormalTok}[1]{#1}
\newcommand{\OperatorTok}[1]{\textcolor[rgb]{0.81,0.36,0.00}{\textbf{#1}}}
\newcommand{\OtherTok}[1]{\textcolor[rgb]{0.56,0.35,0.01}{#1}}
\newcommand{\PreprocessorTok}[1]{\textcolor[rgb]{0.56,0.35,0.01}{\textit{#1}}}
\newcommand{\RegionMarkerTok}[1]{#1}
\newcommand{\SpecialCharTok}[1]{\textcolor[rgb]{0.00,0.00,0.00}{#1}}
\newcommand{\SpecialStringTok}[1]{\textcolor[rgb]{0.31,0.60,0.02}{#1}}
\newcommand{\StringTok}[1]{\textcolor[rgb]{0.31,0.60,0.02}{#1}}
\newcommand{\VariableTok}[1]{\textcolor[rgb]{0.00,0.00,0.00}{#1}}
\newcommand{\VerbatimStringTok}[1]{\textcolor[rgb]{0.31,0.60,0.02}{#1}}
\newcommand{\WarningTok}[1]{\textcolor[rgb]{0.56,0.35,0.01}{\textbf{\textit{#1}}}}
\usepackage{graphicx}
\makeatletter
\def\maxwidth{\ifdim\Gin@nat@width>\linewidth\linewidth\else\Gin@nat@width\fi}
\def\maxheight{\ifdim\Gin@nat@height>\textheight\textheight\else\Gin@nat@height\fi}
\makeatother
% Scale images if necessary, so that they will not overflow the page
% margins by default, and it is still possible to overwrite the defaults
% using explicit options in \includegraphics[width, height, ...]{}
\setkeys{Gin}{width=\maxwidth,height=\maxheight,keepaspectratio}
% Set default figure placement to htbp
\makeatletter
\def\fps@figure{htbp}
\makeatother
\setlength{\emergencystretch}{3em} % prevent overfull lines
\providecommand{\tightlist}{%
  \setlength{\itemsep}{0pt}\setlength{\parskip}{0pt}}
\setcounter{secnumdepth}{-\maxdimen} % remove section numbering
\ifLuaTeX
  \usepackage{selnolig}  % disable illegal ligatures
\fi

\begin{document}
\maketitle

\begin{center}\rule{0.5\linewidth}{0.5pt}\end{center}

\begin{verbatim}
BEGIN ASSIGNMENT
requirements: requirements.R
generate: true
export_cell: false
files: 
  - imgs
\end{verbatim}

In Lab 3, you will learn how to use, edit and create a R Markdown
document (like this one) using RStudio. You should follow the
instructions in this document to complete the assignment. Knit this
document to view the nicely rendered HTML, which can make it easier to
read the questions.

If you need help as you use R Markdown in this lab and others in the
future, consult the following resources:

\begin{itemize}
\tightlist
\item
  \href{https://rmarkdown.rstudio.com/lesson-15.html}{Cheat sheet}
\item
  \href{https://rmarkdown.rstudio.com/docs/}{Home page with guides}
\item
  \href{https://bookdown.org/yihui/rmarkdown/}{Reference book}
\end{itemize}

The below is a code chunk, but instead of using the \texttt{r} engine
we're creating and alert block that will make the question show up with
a blue background in the HTML output. Unfortunately, this creates and
error when exporting to PDF, so it can only be used for HTML.

\begin{verbatim}
BEGIN QUESTION
name: submission
manual: true
\end{verbatim}

\begin{alert alert-info}
\hypertarget{submission-instructions}{%
\subsection{Submission Instructions}\label{submission-instructions}}

rubric=\{mechanics:2\} You receive mark for submitting your lab
correctly, please follow these instructions: -
\href{https://ubc-mds.github.io/resources_pages/general_lab_instructions/}{Follow
the general lab instructions}. -
\href{https://github.com/UBC-MDS/public/tree/master/rubric}{Click here
to view a description of the rubrics used to grade the questions}. -
Push your \texttt{.Rmd} AND \texttt{.html} file to your GitHub
repository for this lab. - The reason for pushing both files is that
\texttt{.Rmd} does not contain the rendered output from running the
cells. If someone is checking out your work there needs to be an HTML
file to view the output, so it is good to get into this habit. -
\texttt{.ipynb} renders nicely on GitHub, which is why we did not
include the HTML file for previous labs. - Upload a \texttt{.html}
version of your assignment to Canvas. - Include a clickable link to your
GitHub repo for the lab just below this cell (it should look something
like this \url{https://github.ubc.ca/MDS-2020-21/DSCI_521_labX_yourcwl}.
\end{alert alert-info}

YOUR LAB GITHUB REPO LINK GOES HERE

\begin{quote}
Solution: Correct link pasted
\end{quote}

\hypertarget{editing-r-markdown-documents}{%
\subsection{Editing R Markdown
documents}\label{editing-r-markdown-documents}}

This document is called an R Markdown document. It is a literate code
document, similar to Jupyter notebooks where you can write code and view
its outputs. To start, let's set our working directory by creating a new
R Project for lab 3.

\hypertarget{text-and-rendering-r-markdown-documents}{%
\subsubsection{Text and rendering R Markdown
documents}\label{text-and-rendering-r-markdown-documents}}

In a R Markdown document any line of text not in a code chunk (like this
line of text) will be formatted using Markdown. Similar to JupyterLab,
you can also use HTML and LaTeX here to do more advanced formatting. To
run a code chunk, you can press the green play button in the top right
corner of the chunk.

\begin{verbatim}
BEGIN QUESTION
name: q1
manual: true
\end{verbatim}

\begin{alert alert-info}
\hypertarget{question-1}{%
\paragraph{Question 1}\label{question-1}}

rubric=\{correctness:1\} Since GitHub does not render HTML files by
default, you need to preface the raw URL to your rendered HTML files
with the following in order to view them online:
\texttt{http://htmlpreview.github.io/?}. Using this approach, include a
clickable link to your rendered HTML file on GitHub under this question.
\end{alert alert-info}

YOUR ANSWER GOES HERE

\begin{quote}
Solution: Correct link pasted
\end{quote}

\begin{verbatim}
BEGIN QUESTION
name: q2
manual: true
\end{verbatim}

\begin{alert alert-info}
\hypertarget{question-2}{%
\paragraph{Question 2}\label{question-2}}

rubric=\{mechanics:1\} Create a new code chunk below using the r
language engine that runs some R code (it does not need to be
complicated, but it should have an output). Ensure that you can
render/knit the document after you add that chunk.
\end{alert alert-info}

YOUR ANSWER GOES HERE

\begin{quote}
Solution: Any valid R code block with an output created.
\end{quote}

\begin{verbatim}
BEGIN QUESTION
name: q3
manual: true
\end{verbatim}

\begin{alert alert-info}
\hypertarget{question-3}{%
\paragraph{Question 3}\label{question-3}}

rubric=\{mechanics:1\} Create a new code chunk, and add a meaningful
name both to this and your previous code chunk. Try using the
pop-up-like menu to navigate between the named code chunks Don't forget
to knit/render the document after you make this change to ensure
everything is still working.
\end{alert alert-info}

YOUR ANSWER GOES HERE

\begin{quote}
Solution: Note that students had a slightly different question here
originally, so they might just have renamed the code block above which
is fine for this year.
\end{quote}

\begin{verbatim}
BEGIN QUESTION
name: q4
manual: true
\end{verbatim}

\begin{alert alert-info}
\hypertarget{question-4}{%
\paragraph{Question 4}\label{question-4}}

rubric=\{mechanics:1,reasoning:1\} Create a new code chunk that uses a
code chunk option. Write out in your own words what the code chunk
option is doing.
\end{alert alert-info}

YOUR ANSWER GOES HERE

\begin{quote}
Solution: Any valid R code block with a chunk option such as
\texttt{echo\ =\ FALSE}.
\end{quote}

\hypertarget{multiple-code-chunk-options}{%
\paragraph{Multiple code chunk
options}\label{multiple-code-chunk-options}}

To have multiple code chunk options you separate them by a comma. For
example, if in addition to suppressing warnings, we want to run the code
but not output the results, then we can add the
\texttt{include\ =\ FALSE} argument to the code chunk after the
\texttt{warning\ =\ FALSE} option.

\begin{verbatim}
BEGIN QUESTION
name: q5
manual: true
\end{verbatim}

\begin{alert alert-info}
\hypertarget{question-5}{%
\paragraph{Question 5}\label{question-5}}

rubric=\{mechanics:1,reasoning:1\} Create a new code chunk that uses at
least two code chunk options. At least one must be different to the ones
mentioned above. Write in your own words what each code chunk option is
doing.
\end{alert alert-info}

YOUR ANSWER GOES HERE

\begin{quote}
Solution: Valid code chunk with two options and correct description.
\end{quote}

\hypertarget{yaml-header-and-document-output-options}{%
\subsubsection{1.5. YAML Header and document output
options}\label{yaml-header-and-document-output-options}}

R Markdown files contains three types of content:

\begin{enumerate}
\def\labelenumi{\arabic{enumi}.}
\tightlist
\item
  Plain text mixed with simple Markdown formatting.
\item
  Code chunks surrounded by ```.
\item
  An (optional) YAML header surrounded by \texttt{-\/-\/-}. You have
  been introduced the first two types of content, but not the third
  (although you probably saw it at the top of this document). The
  (optional) YAML header, which is located at the very top of R Markdown
  files sets some general global parameters, including:
\end{enumerate}

\begin{itemize}
\tightlist
\item
  title
\item
  author
\item
  output
\item
  etc \textbf{Example YAML Header}
\end{itemize}

\begin{verbatim}
---
title: "Reproducible Data Science Report"
author: "Tiffany Timbers"
date: "September 4, 2018"
output: html_document
---
\end{verbatim}

Most important from a workflow perspective is \textbf{output}. Possible
output options include: - \texttt{output:\ html\_document} -
\texttt{output:\ md\_document} - \texttt{output:\ pdf\_document} -
\texttt{output:\ word\_document} -
\texttt{output:\ beamer\_presentation} (beamer slideshow - pdf) -
\texttt{output:\ xaringan::moon\_reader} (xaringan presentation - html)

\begin{verbatim}
BEGIN QUESTION
name: q6
manual: true
\end{verbatim}

\begin{alert alert-info}
\hypertarget{question-6}{%
\paragraph{Question 6}\label{question-6}}

rubric=\{mechanics:1\} Navigate to the YAML header at the very top of
this document and edit it so that you include an \texttt{author}
(yourself) and a \texttt{date} (lab due date). Include what you added
below here as well as a fenced Markdown code block.
\end{alert alert-info}

YOUR ANSWER GOES HERE

\begin{quote}
Solution: It is ok if students only modified the header but did not add
it here since the question was originally different. This might not show
up in the html, so you will need to check the .Rmd in the repo.
\end{quote}

\begin{verbatim}
BEGIN QUESTION
name: q7
manual: true
\end{verbatim}

\begin{alert alert-info}
\hypertarget{question-7}{%
\paragraph{Question 7}\label{question-7}}

rubric=\{mechanics:1\} \textbf{This question previously mentioned to
export the current document (lab3.Rmd). to pdf. This will not work due
to the blue highlight boxes, so please follow these updated instructions
instead.} After you have completed Question 8 and pasted the link to the
HTML output, navigate to the YAML header at the very top of that .Rmd
document and edit it so that the \texttt{output} is
\texttt{pdf\_document}. Then knit/render the document. Note the
different output. Add and commit that rendered \texttt{.pdf} file to the
GitHub repository for this lab and paste a link to that file below this
question.
\end{alert alert-info}

YOUR ANSWER GOES HERE

\begin{quote}
Solution: Correctly exported PDf in the repo and a link here.
\end{quote}

\hypertarget{creating-r-markdown-documents}{%
\subsubsection{Creating R Markdown
documents}\label{creating-r-markdown-documents}}

You can use the ``File'' menu inside RStudio to create new R Markdown
documents by selecting: File \textgreater{} New File \textgreater{} R
Markdown\ldots{} This will bring you to another menu where you can
choose the type of output (don't be afraid to pick something, you can
always change the \texttt{output} type once you have the \texttt{.Rmd}
file).

To create a written report, we generally recommend using the default
\texttt{output:\ html\_document} as it is easier to read than PDF (note
- LaTeX does not render nicely in such documents sadly, so if you are
using a lot of LaTeX then you may want to choose
\texttt{output:\ pdf\_document}). If you want to create an \texttt{.md}
file to publish on GitHub, it is recommend to instead use
\texttt{output:\ github\_document}. To get this from the menu above you
need to navigate to the ``From Template'' option on the left panel and
then select ``GitHub Document (Markdown)''.

To create a presentation we recommend using xaringan. To get a xaringan
presentation from the File \textgreater{} New File \textgreater{} R
Markdown\ldots{} menu, you need to navigate to the ``From Template''
option on the left panel and then select ``Ninja Presentation''. I will
let the package author explain why it is named that way:

\begin{quote}
{[}Sharingan is{]} an R package for creating slideshows with remark.js
through R Markdown. The package name xaringan comes from Sharingan, a
dojutsu in Naruto with two abilities: the ``Eye of Insight'' and the
``Eye of Hypnotism''. A presentation ninja should have these basic
abilities, and I think remark.js may help you acquire these abilities,
even if you are not a member of the Uchiha clan.

\url{https://github.com/yihui/xaringan} When you create R Markdown
documents this way, RStudio gives you a template and some reasonable
defaults to start with to help you get going quickly and easily.
\end{quote}

\begin{verbatim}
BEGIN QUESTION
name: q8
manual: true
\end{verbatim}

\begin{alert alert-info}
\hypertarget{question-8}{%
\paragraph{Question 8}\label{question-8}}

rubric=\{mechanics:6\} 1. Create a new RMarkdown report (a different
file than this one) in the same directory as this Markdown file. Use
\texttt{html\_document} as the \texttt{output}. 2. Create at least two
Markdown text sections (each should have a header) and at least two
separate code chunks in it (these can be really simple). Save the new R
Markdown document and give it a meaningful name. 3. Render/knit the new
R Markdown document to get an \texttt{.html} file. Put the \texttt{.Rmd}
document and the rendered \texttt{.html} file under version control
using Git, and push/upload the file to your GitHub repository for this
homework. Paste a link to these files as your answer below.
\end{alert alert-info}

YOUR ANSWER GOES HERE

\begin{quote}
Solution: Correctly exported html and Rmd in the repo and a link here.
\end{quote}

\begin{verbatim}
BEGIN QUESTION
name: q9
manual: true
\end{verbatim}

\begin{alert alert-info}
\hypertarget{question-9-optional}{%
\paragraph{Question 9 (Optional)}\label{question-9-optional}}

rubric=\{mechanics:1,reasoning:1\} 1. Take the R Markdown report created
in Question 8 and change the output to \texttt{github\_document} and
render it. Put the rendered \texttt{.md} file under version control
using Git, and push/upload the file to your GitHub repository for this
homework. Try to look at the file on GitHub.ubc.ca in your homework
repo? What do you see? How is it rendered?
\end{alert alert-info}

YOUR ANSWER GOES HERE

\begin{quote}
Solution: \texttt{.md} document in the repo. Describe that GitHub
renders the markdown file.
\end{quote}

\begin{verbatim}
BEGIN QUESTION
name: q10
manual: true
\end{verbatim}

\begin{alert alert-info}
\hypertarget{question-10}{%
\paragraph{Question 10}\label{question-10}}

rubric=\{mechanics:6\} 1. Create a presentation using R Markdown. Do
this in a different file than this one but in the same directory as this
Markdown file. Use xaringan (see
\href{https://bookdown.org/yihui/rmarkdown/xaringan.html}{this
tutorial}) as the presentation type. Give this file a meaningful name.
2. Create at least 4 slides. At least two slides must include a code
chunk (these can be really simple). Save the new R Markdown document. 3.
Render/knit the new R Markdown document to get a \texttt{html}
presentation file. 4. Put the new R Markdown document and the rendered
\texttt{.html} file under version control using Git, and push/upload the
file to your GitHub repository for this lab 5. Open the \texttt{.html}
presentation file and print/export to PDF from your browser. Put the PDF
under version control using Git, and push/upload the file to your GitHub
repository for this homework (this is so you have a file in your
homework repo that we can preview). 5. Paste a link to these files
(\texttt{.Rmd}, .\texttt{html}, and \texttt{.pdf}) as your answer below.
\end{alert alert-info}

YOUR ANSWER GOES HERE

\begin{quote}
Solution: Link to the files, and make sure the exported html and PDF
looks reasonable.
\end{quote}

\begin{verbatim}
BEGIN QUESTION
name: q11
manual: true
\end{verbatim}

\begin{alert alert-info}
\hypertarget{optional-question-11}{%
\paragraph{(Optional) Question 11}\label{optional-question-11}}

rubric=\{reasoning:1\} In a paragraph or two, compare and contrast the
use of reproducible tools (e.g., R Markdown and Jupyter) and
non-reproducible tools (Word, Powerpoint, Keynote, etc) for
presentations and reports. Include advantages and disadvantages for
each.
\end{alert alert-info}

YOUR ANSWER GOES HERE

\begin{quote}
Solution: Reproducible tools often makes it easier to collaborate long
term and to track changes made by different people as they faciliate
version control, as well as makes it easier to share and have someone
else recreate your work and pomotes consistent formatting across slides
and presentations. Non-reproduciple tools often have a lower barrier of
entry, especially as people usually have used similar tools before. They
also allow for live collaboration which is sometimes needed, and to
create more fine-grained manual edits to graphics, such as image
placement and aniations in presentations.
\end{quote}

\begin{verbatim}
BEGIN QUESTION
name: placeholder
manual: false
\end{verbatim}

\begin{Shaded}
\begin{Highlighting}[]
\DocumentationTok{\#\# Hidden Test \#\#}
\NormalTok{testthat}\SpecialCharTok{::}\FunctionTok{expect\_true}\NormalTok{(}\ConstantTok{TRUE}\NormalTok{)}
\end{Highlighting}
\end{Shaded}


\end{document}
